\documentclass[man]{apa2}

\usepackage{apacite}

\usepackage{pslatex}
\usepackage{pdfsync}
\usepackage{apacite}
\usepackage{amsmath}
\usepackage{graphicx}
\usepackage{topcapt}
\usepackage{color}
\usepackage{multirow}
%\usepackage{setspace}
%\usepackage[margin=1in]{geometry}
%
%\doublespacing

\title{A Pragmatic Account of the Processing of Negative Sentences}
\author{Ann E. Nordmeyer and Michael C. Frank}
%\affiliation{Department of Psychology, Stanford University}

%\abstract{}

%\shorttitle{Contexts of negative sentences}

\begin{document}
\maketitle

%\begin{abstract}
%Write abstract here.  
%\end{abstract}

%%%%%%%%% INTRO %%%%%%%%% 
\section{Introduction}

START:
COGSCI:
Language is a powerful tool that allows us to describe not only the state of the world as we see it, but also the world as it is not.  If I am a regular at a coffee shop and always order chai, but the shop has run out today, the barista might say ``We don't have any chai today'' when I enter.  Negative sentences are very informative when expectations are violated.



NO CONTEXT WORK: 
COGSCI:
Although negation is critical for communicating many meanings, processing negative sentences can be slow and effortful.  In sentence verification tasks, participants who are asked to evaluate the truth of a sentence describing a picture take significantly longer to evaluate negative sentences compared to positive ones \cite{hclark1972, carpenter1975, just1971, just1976}. In EEG experiments, sentences in which the final noun is semantically unexpected elicit an N400 response, and this response is found even when a negative makes the sentence logically true (e.g.\ ``A robin [is/is not] a truck'')---suggesting that negation is slow to integrate with the rest of the sentence \cite{fischler1983, ludtke2008}.  Similar results have been found in probe-recognition tasks \cite{kaup2003, kaup2006, hasson2006}.  Collectively, this work suggests that processing negative sentences is often difficult.

OLD: 
There is a consistent finding in the literature on negation that participants are slower to evaluate the truth of a sentence when the sentence contains a negative element.  In one such study, conducted by Clark \& Chase (1972), adults participated in a sentence verification task in which participants were asked to compare either affirmative or negative sentences to simple pictures which either matched or did not match the sentences.  Participants were overall slower to respond to negative sentences, such as ``star isn't above plus''.  In addition, an interaction was found between the type of sentence and the truth of the sentence, such that participants were slower to respond to false affirmative sentences than true affirmative sentences, but were faster to respond to false negative sentences than true negative sentences.  \cite{hclark1972}.  The finding that negative sentences incur a greater response time than positive sentences, and the presence of an interaction between the type of sentence and the truth value, is a robust finding that has been replicated many times  \cite{carpenter1975, just1971, just1976}.  The results of studies such as these have generally been taken as evidence for a propositional view of negation, in which a negative sentence is represented as a negative operator operating over a proposition, leading to greater processing time for negative as opposed to positive sentences.  

COGSCI V1:
There is a consistent finding in the literature on negation that participants are slower to evaluate sentences containing a negative element  \cite{hclark1972, carpenter1975, just1971, just1976}.  For example, Clark and Chase (1972) asked participants to compare positive or negative sentences to simple pictures which either matched or did not match the sentences.  Participants were overall slower to respond to negative sentences.  Furthermore, they found an interaction between the type of sentence and the truth of the sentence, with participants responding fastest to true positive sentences and \emph{slowest} to true negative sentences.  This work has been taken as evidence for a propositional view of negation, in which a negative sentence is represented as the denial of an affirmative proposition, leading to greater processing time for negative as opposed to positive sentences. 

The finding that negative sentences take longer to process than positive sentences has been replicated in other paradigms as well.  Electroencephalography studies have shown that semantically unexpected/false sentences such as ``a robin is a truck'' elicit a more negative peak at the N400 compared to sentences such as ``a robin is a bird''.  However, sentences such as ``a robin is not a truck'' produce a greater N400 response than sentences such as ``a robin is not a bird'', suggesting that the negative element ``not'' is not processed immediately (\citeNP{fischler1983}, see also \citeNP{ludtke2008}).  Similar results have been found using probe-recognition tasks.  For example, the sentence ``The door was open'' primes recognition of  a matching picture (an open door) compared to a mismatching picture (a closed door) after 750 ms delay, whereas negative sentences such as ``The door was not open'' do not prime the matching picture (a closed door) until a 1000 ms delay.  (\citeNP{kaup2006}, see also \citeNP{kaup2003, hasson2006}).  Although there is some disagreement about the nature of these representations (e.g. as the denial of a proposition (see \citeNP{hclark1972}), or a mental model of the negated and actual state of affairs (see \citeNP{kaup2003}), this work collectively suggests that negative sentences are more difficult to process than positive sentences.  










CONTEXT WORK:
 COGSCI:
There is a critical difference, however, between evaluating a sentence in the lab and comprehending speech in the real world. According to Grice's Cooperative Principle \cite{grice1975}, speakers should produce utterances that are truthful, relevant, and informative.  Negative sentences presented without context violate this principle.  If the barista says ``we don't have chai today'' to a customer who always orders coffee, this utterance would be neither relevant nor informative.  In general, negations are produced when there is some expectation that the speaker wishes to reverse.  

Congruent with this Gricean account, a number of studies have shown that a supportive context mitigates the processing cost of negation \cite{wason1965, glenberg1999, ludtke2006, nieuwland2008, dale2011}. Some contexts are more effective than others at reducing processing demands. For example, contexts that explicitly mention a negated characteristic \cite{ludtke2006} or that present the negation within a dialogue \cite{dale2011} elicit faster reaction times, perhaps because the negation is more informative. But although these findings are congruent with the idea that pragmatic expectations are the source of negation's processing cost, they do not directly test that hypothesis.  The goal of our current work is to make such a test.


OLD:
There is a critical difference between evaluating sentences in a laboratory setting, where sentences are presented without any communicative context, and comprehending speech in the real world. According to Grice's cooperative principle \cite{grice1975}, speakers should produce utterances that are appropriately informative, relevant, and unambiguous.  However, in many cases producing a negative sentence can be uninformative, ambiguous, and irrelevant.  If my friend asks me what I did this week, and I respond ``I bought a car!'', this piece of new information is informative and relevant to the conversation.  However, if I respond ``I didn't buy a car this week'', my contribution to the conversation is neither relevant nor informative.  The fact that unsupported negative utterances are generally rated as more ambiguous than unsupported affirmative utterances makes intuitive sense, and has been documented experimentally \cite{glenberg1999}.  In general, negative utterances are produced in contexts where there is some \emph{expectation} that the speaker wishes to negate.  For example, if my friend knows that I have been in the market for a new car, then she may be asking about my week because she expects I bought a new car, and my statement ``I didn't buy a car'' becomes relevant and informative.  

Noting that denials are generally produced in response to a violation of expectations, Wason (1965) designed a study to examine whether pragmatic constraints such as context would affect how participants responded to negative sentences.  Participants viewed stimuli consisting of 8 colored dots, in which 7 dots were one color and 1 dot was a different color.  Participants were asked to describe the stimuli, and then evaluate positive and negative sentences about the stimuli.  Wason found that when participants' descriptions of the stimuli highlighted the fact that one of the dots was an exception to the rule, they were faster to evaluate negative sentences about the ``odd'' dot.  Additional work has shown that when participants read sentences that set up a supportive context, reading times for negative sentences are reduced \cite{glenberg1999, ludtke2006}, and mouse trajectories to evaluate the truth of negative sentences are smoother and less discrete \cite{dale2011}.  Children are also sensitive to context effects when processing negative sentences, responding faster to respond to negative sentences about a toy that is different from other toys in the group \cite{devilliers1975}, and producing more spontaneous negatives to describe a stimulus that lacks an obvious characteristic \cite{watson1979}.  This work suggests that pragmatics and contextual support are particularly important to the ease at which negative sentences are processed.   

COGSCI V1:
There is a critical difference, however, between evaluating sentences presented without any context in a laboratory setting and comprehending speech in the real world. According to Grice's cooperative principle \cite{grice1975}, speakers should produce utterances that are appropriately informative, relevant, and unambiguous.  Negative sentences presented without any contextual support often violate this principle.  If my friend asks me what I did this week, and I respond ``I didn't buy a car'', my contribution to the conversation is neither relevant nor informative.  The fact that unsupported negative utterances are generally rated as more ambiguous than unsupported affirmative utterances has been documented experimentally \cite{glenberg1999}.  In general, negative utterances are produced in contexts where there is some \emph{expectation} that the speaker wishes to negate.  For example, if my friend knows that I have been in the market for a new car, then she may be asking about my week because she expects I bought a new car, and my statement ``I didn't buy a car'' becomes relevant and informative.  

Noting that denials are generally produced in response to a violation of expectations, Wason (1965) designed a study to examine whether pragmatic constraints such as context would affect how participants responded to negative sentences.  Participants viewed stimuli consisting of 8 colored dots, in which 7 dots were one color and 1 dot was a different color.  Participants were asked to describe the stimuli, and then evaluate positive and negative sentences about the stimuli.  Wason found that when participants' descriptions of the stimuli highlighted the fact that one of the dots was an exception to the rule, they were faster to evaluate negative sentences about the ``odd'' dot.  Additional work has shown that when participants read sentences that set up a supportive context, reading times for negative sentences are reduced \cite{glenberg1999}, and false positive and false negative sentences show similar N400 responses when the negative sentences are pragmatically licensed \cite{nieuwland2008}.  Children are also sensitive to context effects when processing negative sentences, responding faster to respond to negative sentences about a toy that is different from other toys in the group \cite{devilliers1975}.

Some researchers have found that some contexts are more effective than others at facilitating the processing of negative sentences.  In one such study, contexts which explicitly mentioned or strongly implied the negated characteristic were more effective at reducing reading times of negative sentences compared to contexts which did not mention the relevant characteristic \cite{ludtke2006}.  Critically, these contexts did not have a significant effect on the reading times for positive sentences.  In addition, a mouse-tracking study conducted by Dale and Duran (2011) \nocite{dale2011} found that enhanced contexts lead to significantly smoother mouse trajectories when selecting whether a sentence was true or false, but simple one-sentence contexts did not have an effect.  This work suggests that there is some graded effect of context, with some types of context having a greater effect on negation processing than others, though few studies have thoroughly examined the effect of the strength of the context.











MODEL SETUP: 
COGSCI:
We propose that negative sentences are more informative in contexts that set up a strong expectation that is violated. If the processing cost of negation is pragmatic, then more informative negative sentences should elicit smaller reaction times. How should we quantify informativeness in context? Recent modeling work quantifies pragmatic reasoning in simple experimental contexts \cite{frank2012,goodman2013}. The assumption underlying this work is that speakers are informative---they will produce utterances that will pick out smaller subsets of the context, leaving as little ambiguity as possible for the listener.  We use this definition of informativeness to provide a quantitative interpretation of our hypothesis.

To link informativeness---as computed in a probabilistic model---to reaction time, we assume that reaction time is proportional to \emph{surprisal}. Surprisal is an information-theoretic measure of the amount of information carried by an event (in this case, an utterance in some context) based on its probability. Surprisal has been used effectively to predict reaction times from probabilistic models \cite{levy2008}; this work provides inspiration for our current model. 

OLDER COGSCI:
What is the mechanism by which context influences the processing of negation?  We propose that context affects the informativeness of a negative utterance, such that negative sentences are more informative when presented in contexts that set up a strong expectation that is violated.  Recent work by Frank et al. \cite{frank2012} has attempted to quantify the pragmatic inferences that adults make when playing simple ``language games''. The assumption underlying the predictions made by these authors is that speakers are informative - that is, they will produce utterances that will pick out smaller subsets of the context, leaving as little ambiguity as possible for the listener.  If listeners predict that speakers will choose more informative sentences, then they should process negative sentences that are highly informative in context faster than negative sentences that are less informative in context.  This hypothesis draws on the idea of \emph{surprisal}, a information-theoretic measure of the amount of information contained in an utterance.  This theory proposes that expectations about what utterance a speaker will produce are related to a listener's speed to evaluate that sentence  (see\citeNP{levy2008} for a syntactic theory of surprisal).  We draw on this theory in the realm of pragmatics, developing a pragmatic theory of surprisal to explain the effect of context on the processing of negative sentences.  









END:
COGSCI:
We test the hypothesis that pragmatic surprisal explains the processing cost of negative sentences. Study 1 measures this processing cost, replicating previous findings that context facilitates the processing of negation.  Study 2 investigates the effect of the strength of the context by parametrically varying the base rate of a negated feature.  We compute the surprisal of sentences in these contexts, and find that a model of pragmatic informativeness predicts the relationship between context and reaction time.  These results support the idea that context affects negative sentence processing by modulating listeners' expectations. 





THE REST IS ALL FROM COGSCI CAMERA-READY VERSION:

\section{Study 1: Context vs. No Context}

To test whether non-linguistic contextual expectations alleviate the processing cost of negative sentences, we constructed a simple sentence verification task based on \citeA{hclark1972}.  Previous studies of the relationship between context and negation have required participants to actively engage with the context, either by describing pictures \cite{wason1965} or reading sentences \cite{glenberg1999}.  Here, participants passively viewed a visual context, eliminating linguistic confounds in previous work.  

\subsection{Method}

\subsubsection{Participants}

We recruited 100 participants to participate in an online experiment through the Amazon's Mechanical Turk (mTurk) website.\footnote{Previous work has shown that mTurk is an effective tool for collecting RT data \cite{crump2013}.}  Participants ranged in age from 18-65; 63 were male and 37 female.  We restricted participation to individuals in the United States. We paid participants 30 cents to participate, which took approximately 5 minutes to complete.  

\subsubsection{Stimuli}

\begin{figure}[t]
\begin{center} 
\includegraphics[width=3.25in]{figures/negatron_trialfig2.pdf}
\caption{\label{fig:trial} An example trial, consisting of two separate slides (shown sequentially): a context slide and a trial slide for a true negative trial. }
\vspace{-5mm}
\end{center} 
\end{figure}

Twenty-eight trial items were created in which a character was shown holding either two of the same common, recognizable objects (e.g.\ two apples), or holding nothing.  On each trial a sentence of the form ``[NAME] [has/has no] [ITEM]'' was written.  Half of the sentences were positive and half were negative, and they were paired with pictures such that half were true and half were false.  The experiment was fully crossed, with participants receiving seven true positive, seven false positive, seven true negative and seven false negative sentences in a randomized order over the course of the study.  

Participants were randomly assigned to the ``no context'' condition or the ``context'' condition.  Participants in the no context condition saw a blank screen with a fixation cross before each trial, while participants in the context condition viewed a context slide.  The context slide showed three characters, each holding the same two identical items.  The characters all differed from the trial character and from each other in hair and shirt color.  A sentence instructed participants to ``Look at these [boys/girls]!'' (Fig.\ \ref{fig:trial}).  


\subsubsection{Procedure}
Participants were first presented with an instructions screen which described the task and informed them that they could stop at any time.  Once they accepted the task, they were given eight positive sentence practice trials with feedback about incorrect responses. 

In each trial, participants saw a context (3s) and then a picture and a sentence. They were asked to read the sentence and respond as quickly and accurately as possible with a judgment of whether it was true or false when applied to the picture.  We recorded reaction times for each trial, measured as the time from when the picture and sentence were presented to the moment when the response was made.

\subsubsection{Data Processing}
We excluded from analysis 6 participants who did not list English as their native language, 7 participants for having participated in a previous pilot study, and 4 participants for having an overall accuracy of below 80\%.  Thus, data from a total of 83 participants were analyzed.  We also excluded trials with RTs greater than 3 standard deviations from the log-transformed mean.  

\subsection{Results \& Discussion}

\begin{figure}
\begin{center} 
\includegraphics[width=3.25in]{figures/study1_linegraph.pdf}
\caption{\label{fig:e1line} Reaction times for each trial type across different conditions.  Responses to true sentences are shown on the left, and false sentences are shown on the right.  Negative sentences are shown in grey, and positive sentences in black.  Error bars show 95\% confidence intervals.}
\end{center} 
\end{figure}

Negative sentences were difficult to process when presented without context; in context, this effect disappeared (Fig.\ \ref{fig:e1line}).  This result is congruent with previous work on sentence verification, which has also found a main effect of negation \cite<e.g.>{hclark1972} and with work examining the role of context in negation \cite<e.g.>{wason1965, nieuwland2008, dale2011}.  

To examine the reliability of these findings, we fit a linear mixed-effects model to participants' reaction times.  We examined the interaction between sentence type, truth value, and context on reaction times.\footnote{All mixed-effects models were fit using the lme4 package in R version 2.15.3.  The model specification was as follows: \texttt{RT $\sim$ sentence~$\times$~truth~$\times$~context + (sentence~$\times$~truth~\textbar~subject) +  (sentence~$\times$~truth~\textbar~item)}.  Significance was calculated using the standard normal approximation to the $t$ distribution \cite{barr2013}. Data and analysis code can be found at \href{http://github.com/anordmey/cogsci14_negatron}{http://github.com/anordmey/cogsci14\textunderscore negatron}} Results of this model show a main effect of truth value, with significant faster reaction times for true sentences compared to false sentences ($\beta= -196$, $p< .001$).\footnote{Coefficient weights are interpretable in milliseconds.}  Although there was no main effect of negation across both conditions, there was an interaction between sentence type and truth value ($\beta= 260$, $p< .001$), replicating the finding that participants respond fastest to true positive sentences but slowest to true negative sentences \cite{hclark1972}.  Critically, there was a significant 3-way interaction between context condition, sentence type, and truth value ($\beta= -227$, $p< .01$), suggesting that this interaction was primarily driven by the slow RTs for true negative sentences in the no context condition.  

To understand why context had the strongest effect on true negative sentences, consider what a true negative trial looks like in the no context condition.  These are trials in which the participant has no expectation about what the character might be holding, because no context was provided to set up such an expectation.  The participant would then see a picture of an empty-handed boy with the sentence ``Bob has no apples.''  These types of trials likely cause participants to falter because there is no reason for ``apples'' to be mentioned at all.  However, when a participant first views a context such as the one in Fig.\ \ref{fig:trial}, they can form an expectation that boys typically have apples.  Now, when participants see a boy with no apples, a sentence such as ``Bob has no apples'' makes sense.

Study 1 contributes to a body of evidence suggesting that negative sentences are more felicitous when they negate an expectation, and that such expectations can be set up by an appropriate context.  In Study 2, we examine how systematically manipulating the strength of the context might produce changes in reaction times by altering the expectations created by the context.  

\section{Study 2: Varying strength of context}

Should all contexts be equally helpful in processing negation? In Study 2, we parametrically manipulated the strength of the context.  Participants saw contexts consisting of either three (Study 2a) or four (Study 2b) characters in which some subset of the characters were holding the target item.  If the context gives participants a glimpse into the ``world'' that each trial exists in, this represents a small sample of the base rate of what the characters in this world look like.  By manipulating this base rate, we can change peoples' expectations about the trial character.  If the differences in reaction times between the no context and the context condition in Study 1 are due to the relative informativeness of the negative utterance based on the context, we should expect to see a relationship between the strength of the context and reaction time. 

\subsection{Method}

\subsubsection{Participants} 

We again recruited participants from mTurk, 200 in 2a (129 male, 71 female) and  400 in 2b (205 male, 195 female), ages 18 -- \textgreater65. We again restricted participation to individuals in the US and paid 30 cents for this 5 minute study.  

\subsubsection{Stimuli}

Study 2a used the same 28 trial items and sentence types as those used in Study 1.  A between-subjects factor determined what type of context participants saw.  Context conditions showed $\frac{0}{3}$, $\frac{1}{3}$, $\frac{2}{3}$, or $\frac{3}{3}$ of the characters holding objects.  Trial stimuli were identical to those in Study 1.  

Study 2b used 48 items.  The contexts were the same as in Study 2a, except that each context contained 4 characters and there were therefore 5 context conditions ($\frac{0}{4}$, $\frac{1}{4}$, $\frac{2}{4}$, $\frac{3}{4}$ or $\frac{4}{4}$).  

\subsubsection{Procedure}
 The procedure for Study 2a was identical to that of Study 1, with participants randomly assigned to condition.   In Study 2b, participants were given 4s (instead of 3s) to view the context before the experiment advanced.  This latency was changed to give participants more time to look at the slightly larger contexts; the procedure was otherwise identical.
 
 \subsubsection{Data Processing}
We excluded 35 participants who did not list English as their native language (9 in 2a and 16 in 2b), 24 participants for participating in a previous iteration of the experiment (3 in 2a and 21 in 2b), and 35 participants for having an overall accuracy below 80\% (11 in 2a and 24 in 2b).  Thus, we analyzed data from a total of 177 participants in Study 2a and 339 participants in Study 2b. As in Study 1, we only analyzed correct trials and excluded trials with RTs greater than 3 SDs from the log-transformed mean. 

Because we were interested in the effect of context, results from these two studies were combined and analyzed together with context condition re-coded as a continuous variable by calculating the proportion of people in each context condition who had a target item (e.g.\ the $\frac{1}{3}$ condition in Study 2a was recoded as .33). 

\subsection{Results and Discussion}

\begin{figure}[t]
\begin{center} 
\includegraphics[width=3.25in]{figures/study2a_linegraph.pdf}
\includegraphics[width=3.25in]{figures/study2b_linegraph.pdf}
\caption{\label{fig:e2line} Reaction times for each trial type across different conditions. Responses to true sentences are shown on the left, and false sentences are shown on the right.  Negative sentences are shown in grey, and positive sentences in black.  Data for Study 2a (3-person contexts) are shown above, and data for Study 2b (4-person contexts) are shown below.  The context condition is notated by a fraction representing the number of characters in the context who held target items. Error bars show 95\% confidence intervals.  }
\end{center} 
\end{figure}

As the proportion of target items in the context increased, reaction times tended to decrease, particularly for negative and false sentences, supporting our hypothesis (Fig.\ \ref{fig:e2line}).  Unexpectedly, reaction times increased slightly when all characters in the context had target items, resulting in a U-shaped relationship between context and RT.  

We fit a linear mixed-effects model to reaction times in response to sentences.  We examined the interaction between sentence type, truth value, and context on reaction times.\footnote{The model specification was as follows: \texttt{RT $\sim$ sentence~$\times$~truth~$\times$~context + (sentence~$\times$~truth~\textbar~subject) +  (sentence~$\times$~truth~\textbar~item)}.}  As in Study 1, we found a significant effect of truth value, with significantly faster reaction times for true sentences compared to false sentences ($\beta= -154$, $p< .001$).  Although there was not a significant main effect of negation, there was a significant interaction between sentence type and truth value, such that the difference between true positive and true negative was greater than the difference between the two types of false sentences ($\beta= 159$, $p< .001$).  There was also a linear effect of context, such that as the proportion of people with the target item in the context increased, reaction times decreased ($\beta= -197$, $p< .001$).  As before, there was a significant 3-way interaction between context, sentence type, and truth value, such that the linear effect of context was most striking in true negative sentences ($\beta= -141$, $p< .001$).

Responses in the $\frac{3}{3}$ and $\frac{4}{4}$ conditions suggest that the relationship between context and RT is not linear (Fig.\ \ref{fig:e2line}).  We added a quadratic term to our model to test for this nonlinear effect of context ($\beta= 610 $, $p< .001$).  The quadratic model fit our data significantly better in a likelihood comparison test ($\chi^{2}(1) =80.59$, $p<.001$).  

Quantitative manipulation of the strength of the context resulted in systematic changes in the processing cost of negation, particularly for true negative sentences.  This finding is consistent with our initial hypothesis: As the proportion of people in the context with the target item increases, describing the trial picture as \emph{not} having that target item becomes more informative.  That is, the more people in the context who have apples, the more we expect a person with nothing to be described as ``a boy with no apples.'' 


SHOULD I ADD STUDY 3 BACK IN??
%\section{Study 3: Measuring Listener Expectations}
%In studies 1 and 2, we demonstrated that a simple visual context can facilitate the processing of negative sentences, and that there is a linear (quadratic?) relationship between the strength of the context and the effect on reaction time to evaluate negative sentences.  Contexts that set up a strong expectation lead to faster reaction times to process a negative sentence when that expectation is violated.  Our prediction was based on previous work demonstrating that participants expect speakers to be informative when speaking \cite{frank2012}.  If everyone in the context has a specific feature, and the trial character is lacking that feature, it is highly informative to describe the trial character in terms of the negation of the expected feature.  However, although the results of Studies 1 and 2 support this interpretation, we do not know for sure whether participants expect speakers to use negation in these contexts.  Study 3 attempts to directly measure participants' expectations of how a speaker would describe the pictures seen in Studies 1 and 2, depending on the context.

%Our hypothesis here draws on the idea of \emph{surprisal}, a information-theoretic measure of the amount of information contained in an utterance.  Previous work has proposed that the processing difficulty of a word is the surprisal of that word, calculated as the -log of the probability of the word occurring in a given syntactic and semantic context \cite{levy2008}.  This theory proposes that expectations about what utterance a speaker will produce are related to a listener's speed to evaluate that sentence.  We draw on this theory in the realm of pragmatics, developing pragmatic theory of surprisal to explain the effect of context on the processing of negative sentences.  

%To test this theory, we conducted a study to evaluate participants' expectations of how a speaker would describe a picture in a given context.  Participants were introduced to an imaginary speaker who was described as either ``an honest guy who says things that are plausible and true'' or ``a tricky guy who says things that are plausible but false''.  This was originally set up to test participants expectations about false sentences; however, participants in the ``tricky'' condition struggled to understand this task, and results from this condition are not presented here.  Participants were then shown a set of three or four characters that the speaker ostensibly also saw (the contexts from the previous studies), and then shown a single character and asked to bet whether the speaker would use different sentences to describe the picture.  This allowed us to calculate the surprisal of positive and negative sentences based on participants' expectations about what sentences a speaker would use to describe these pictures, and determine if a relationship exists between the pragmatic surprisal of a sentence and reaction times to evaluate those sentences.  

%\subsection{Method}

%\subsubsection{Participants}
%280 participants were recruited to participate in an online experiment through Amazon's Mechanical Turk website.  Participants ranged in age from 18 - 65+, with the majority of participants being between 25 and 35 (109 participants).  167 participants were male and 113 were female.  Participation was restricted to individuals in the United States, and participants who indicated that their primary language was something other than English were excluded from analysis.  Participants were paid 30 cents to participate in the study, which took approximately 5 minutes to complete.  

%\subsubsection{Stimuli}
%A subset of 12 items out of the items used in Studies 1 and 2 were used for this study.  As in the previous studies, these items consisted of a context which presented either three (Study 3a) or four (Study 3b) characters who were either holding two of the same item or holding nothing, and a trial character who was presented alone holding either two items or holding nothing.  

%On each trial, participants were given four sentences to bet on.  The four sentences were always a positive sentence involving the target item, a negative sentence involving the target item, a positive sentence involving an alternative item, and a negative sentence involving an alternative item.  This meant that for pictures of a person holding a target item (e.g. a picture of a boy with apples), the two true sentence options were the positive-item sentence and the negative-alternative sentence (e.g. ``A boy with apples'' and ``A boy with no cookies'').  For pictures of a person holding nothing, the two true sentence options were the negative-item sentence and the negative-alternative sentence (e.g. ``A boy with no apples'' and ``A boy with cookies'').   

%\subsubsection{Procedure}
%As in Studies 1 and 2, participants were first presented with an instructions screen which described the task and informed them that they would be participating in a psychology experiment and could stop at any time.  Once participants accepted the task, they were introduced to the speaker, Joe, who was described as ``an honest guy who says things that are plausible and true''.  They were told that they would see different pictures, and their job was to decide how likely Joe is to use different sentences to describe the picture.  Participants were told that they could bet on more than one sentence, but that their bets must sum to 100.  

%Participants were then given two trials to practice betting.  Practice trials always showed a person with items, and did not include context slides.  Participants were reminded of the task on each trial.  During practice trials, participants were only given positive sentences to choose from, and only one of the sentences was true.  If participants bet on a false sentence, they were reminded that the speaker is an honest person who says things that are true.  This feedback was not given during test trials.

%Experiment trials differed from practice trials in that participants first viewed a context slide similar to the context slides presented in Studies 1 and 2.  Participants saw three (Study 3a) or four (Study 3b) characters with either none, one, two, three, or four of the characters holding target items with the rest of the characters holding nothing.  Beneath the characters, participants were told ``Joe sees these [boys/girls]''.  The context screen was displayed for four seconds and then the experiment automatically proceeded to the test trial for that item.  

%Test trials showed a single boy or girl either holding the same items or holding nothing.  Under the picture were four sentences (positive-item, negative-item, positive-alternative, and negative-alternative).  Participants were told to place bets on whether Joe would use these sentences to describe this picture, and were reminded that their bets must sum to 100.  

%\subsection{Results}
%\begin{figure}
%\begin{center} 
%\includegraphics[width=3.25in]{figures/speakerstudy_comparison.pdf}
%\caption{\label{fig:e3plot} A comparison of surprisal from Study 3 (calculated as the -log of the mean bet on each sentence type) and reaction times from Study 2.  Error bars represent the 95\% confidence intervals.  }
%\end{center} 
%\end{figure}

%Study 3 was designed to test the effect of context on people's expectations of how a speaker would describe a picture.  We predicted that there would be a correlation between participants predictions of what a speaker would say to describe a picture, and reaction times to evaluate the predicted sentences.  

%Eleven participants who listed a language other than English as their native language were excluded from analysis.  Forty-six additional participants were excluded for having participated in a previous iteration of the experiment. Thus, data from a total of 223 participants were analyzed.  

%In each trial, participants had the ability to bet on four possible sentences: a positive sentence about the target item, a negative sentence about the target item, a positive sentence about an alternative item, and a negative sentence about the alternative item.  Any bets on sentences that were logically false were excluded; very few participants bet on false sentences (the average participant allocated less than 5\% of their bets to false sentences).  This eliminated all bets to alternative-positive sentences (e.g. ``A boy with cookies'' when the target item was apples), because these sentences were always false.  We also eliminated all bets to alternative-negative items (e.g. ``A boy with no cookies'' when the target item was apples).  Although these sentences were sometimes true (e.g. when the trial picture showed a boy with nothing), we were interested in seeing how participants bets on item negative sentences changed as the context changed.  Thus, we only analyzed bets to item-positive sentences when the trial picture showed a character holding the target items (e.g. bets on ``A boy with apples'' when the picture showed a boy holding apples) and bets to item-negative sentences when the trial picture showed a character holding nothing (e.g. bets on ``A boy with no apples'' when the picture showed a boy holding nothing).  These were analogous to the True Positive and True Negative sentences in Studies 1 and 2, allowing us to compare expectations about what a speaker would say to participants' reaction times in Study 2.  

%We used participants' expectations about how a speaker would describe a given picture to calculate surprisal, an information-theoretic measure that previous work has suggested is related to the processing difficulty of a word \cite{levy2008}.  Surprisal is calculated as the -log of the probability of a word given the context.  We calculated the probability of a word given the context by taking the mean bet for each sentence type (true negative and true positive, as described above) for each context condition, and took the -log of these mean bets to compute the surprisal of each sentence type for each context condition.  We then compared this measure of surprisal to the mean reaction times for true positive and true negative sentences for each context condition, as measured in Study 2.  A graph of this comparison can be seen in Figure \ref{fig:e3plot}.  There was a significant correlation between surprisal and reaction time in these data ($r=.82$, $p<.001$).  

%\subsection{Discussion}

%In Studies 1 and 2, we found that including a visual contexts facilitates the processing of negative sentences, and that this effect is modulated by the strength of the expectation set up by the context.  As the proportion of people with a target item in the context increases, the reaction time to process a negative sentence that negates that target property decreases.  We postulated that this effect is due to the ways that context changes expectations about what a speaker might say to describe a picture.  If you see a context in which the majority of boys are holding apples, and then see a boy holding nothing, it makes sense to predict that the boy holding nothing might be described as a ``boy with no apples''.  However, if you see a context in which nobody is holding anything, it would be odd to describe another person holding nothing as ``a boy with no apples'', because you have no reason to expect that he \emph{would} have apples.  Our prediction was that these expectations, calculated as the surprisal of seeing a picture described using a certain sentence, would be positively correlated with reaction times from previous studies.  Study 3 confirmed this hypothesis.  

%Why is it that different contexts influence our predictions of how a speaker might describe a picture?  Our initial hypothesis was built on work by Frank and Goodman (2012), who created a model of language comprehension based on the Grice's Maxim of Quantity, which dictates that speakers should choose utterances that are maximally informative.  This model demonstrated that listeners expect speakers to produce utterances that are highly informative.  With respect to this work, the utterance ``A boy with no apples'' is highly informative if it refers to a boy with nothing in a world where every other boy has apples, because this utterance uniquely identifies the intended referent.  However, in a world where none of the other boys are holding apples, the sentence ``A boy with no apples'' is uninformative, because it refers to every other boy in this hypothetical world.  In the next section, we test whether a model of informativeness can predict processing times for true positive and true negative sentences.  

\section{Model}

Studies 1 and 2 show that a simple visual context can facilitate the processing of negation, with contexts that set up a strong expectation leading to faster RTs for negative sentences.  We hypothesized that this effect was driven by the expectation that speakers are informative \cite{grice1975,frank2012}: If everyone in a context has a specific feature, and the target character is lacking that feature, it is highly informative to describe the target character in terms of the negation of the expected feature. In contrast, if no one has a feature, it's pragmatically odd to negate it. In this section, we formalize these intuitions. Due to the Gricean nature of the intuition---which lead us to consider a truthful speaker as well---we focus here on predicting the processing of true sentences.   

\subsection{Model 1: Utterance Surprisal}

We modeled the behavior of participants in our experiments by assuming that reaction time is proportional to the surprisal of the utterance $w$, given the context $C$ and the speaker's intended referent $r_S$ (following \citeNP{levy2008}):

\begin{equation}\label{eq:surprise}
RT \sim -\log(P(w| r_s, C)).
\end{equation}

\noindent We then define the probability of the utterance as proportional to its utility (following \citeNP{frank2012}):

\begin{equation}\label{eq:pw1}
P(w | r_s, C) \propto  e^{U(w;r_s,C)},
\end{equation} 

\noindent This utility is defined as the informativeness of $w$ minus its cost $D(w)$:

\begin{equation}\label{eq:utility}
U(w;r_s,C) = I(w;r_s, C) - D(w).
\end{equation}

\noindent Informativeness in context is calculated as the number of bits of information conveyed by the word. We assume that $w$ has a uniform probability distribution over its extension in context (e.g.\ ``boy with apples'' applies to any boy who has apples, leading to a probability of $1/|w|$ of picking out each individual boy with apples) :

\begin{equation}\label{eq:info}
I(w;r_s, C) = -(-\log(|w|^{-1})).
\end{equation}

\noindent The cost term $D(w)$ can then be defined in any number of ways; in this model we define it as the number of words in the utterance multiplied by a cost-per-word parameter.  Note that in our experiment, the negative sentences always have exactly one word more than the positive sentences. 

We created a sparse vocabulary which represented possible words to describe the characters.  This included the target utterance (e.g.\ ``apples'' and ``no apples''), as well as words that were uniformly true or false of all characters. Combining Equations \ref{eq:pw1}--\ref{eq:info}, and normalizing Eq.\ \ref{eq:pw1} over all possible words in the vocabulary $V$, we have:

\begin{equation}\label{eq:pw2}
P(w | r_s, C) = \frac{ e^{\log(|w|^{-1}) - D(w)}} {\sum_{w' \in V}{e^{\log(|w'|^{-1}) - D(w')}}}.
\end{equation}

\noindent Combining Eq. \ref{eq:surprise} with Eq. \ref{eq:pw2}, this model predicts that as the number of e.g.\ boys with apples in the context increases, the informativeness of the negative sentence ``Bob has no apples'' increases, because it selects an increasingly smaller subset of the context. Highly informative sentences will have high probability, hence lower surprisal and faster RTs. 

We fit this model to data from Study 2a, with cost = 3 (Table 1).  When the model was fit to the combined data from Studies 2a and 2b, the cost-per-word parameter remained the same (Fig.\ \ref{fig:model1_sims}).  This model accounted for a substantial amount of variance in participant reaction times from Study 2 ($r=.76$, $p<.001$).  Nevertheless, the model fails to capture the U-shaped relationship seen in Study 2; specifically, it underestimates the surprisal of $\frac{0}{3}$ and $\frac{0}{4}$ contexts for positive sentences, and $\frac{3}{3}$ and $\frac{4}{4}$ contexts for negative sentences.In these trials, participants may have found the target picture surprising regardless of the sentence that they read. For example, in $\frac{0}{3}$ and $\frac{0}{4}$ contexts followed by a true positive trial, participants saw several boys with nothing, and then saw a boy holding something.  

\subsection{Model 2: Utterance and Referent Surprisal}

To account for reaction time related to seeing the target picture, we included the surprisal of the referent $r_S$ as well as the surprisal of the utterance $w$ given the referent. We estimated the probability of seeing the referent via the count of the target property in the context, smoothed with a parameter $\lambda$:

\begin{equation}\label{eq:pp}
P(r_S | C) =  \frac{\# Matching People  + \lambda}{\# Total People + 2\lambda}
\end{equation}

We then added $-\log(p(r|C))$ (Eq.\ \ref{eq:pp}) to $-\log(p(w|r_s,C))$ (Eq.\ \ref{eq:pw2}), resulting in:

\begin{equation}\label{eq:total}
RT \sim - \log(P(w|r_s, C)) - \beta \log(P(r_S|C)).
\end{equation}

\begin{table}[t]
\caption{\label{tab:modelcorrs} Model parameters and correlations between model predictions and reaction times, for both Model 1 (Utterance surprisal only) and Model 2 (Utterance and referent surprisal.  Parameters are either fit to Study 2a only or to both studies, as indicated.}
\begin{center}
\small\addtolength{\tabcolsep}{-2pt}
\begin{tabular}{ r| r|  r  r  r  r} 
  \bf{Model} & \bf{Data} & \bf{cost} & \bf{$\lambda$} & \bf{$\beta$} & \bf{$r$}  \\ \hline        
1 (Utt only) &  Study 2a  & 3 & & & .84\\     
& Study 2b (2a params) & 3  & & & .71\\
  & Both (2a params) &  3 & &  & .76 \\
  & Both & 3 &   &   & .76 \\ \hline
2 (Utt + Ref)  & Study 2a & .5 & .1 & .3 & .95\\     
& Study 2b (2a params) & .5  & .1 & .3 & .86\\
  & Both (2a params) &  .5 & .1 & .3 & .89\\
  & Both & .4 &   .2 &  .4 & .90\\ 
\hline
\end{tabular}
\end{center}
\end{table}

\noindent Note that this formulation is quite similar to a model which accounts for the prior probability of the referent $p(r_S)$; the only difference is our use of a weight $\beta$ to adjust the different effects of these two probabilities.  

Consider the example in Fig.\ 1, in which you see three boys with apples and then a boy with no apples.  The sentence ``Bob has no apples'' is highly probable---and thus low surprisal---in this context, because it uniquely identifies the target character (Eq.\ \ref{eq:pw2}).  In the current model, however, we must also calculate the surprisal of seeing the target character (i.e. the referent).  In this example, the referent surprisal is high, because the probability of seeing a boy with no apples in this context is low (Eq. \ \ref{eq:pp}).  

We again fit this model to data from Study 2a and compared model predictions to data from Study 2b as well as the combined data from 2a and 2b (Table 1).  Using the parameters fit to Study 2a, the model accounted for a substantial amount of variance in participant reaction times from Study 2 ($r=.89$, $p<.001)$.  We also fit the model to the combined data from Study 2, which resulted in similar parameter values (Table 1), and continued to account for a substantial amount of the variance in RTs ($r=.90$, $p<.001$; Fig.\ \ref{fig:model1_sims}).   

\begin{figure}[t]
\begin{center} 
\includegraphics[width=3.25in]{figures/model1_comparison.pdf}
\includegraphics[width=3.25in]{figures/model2_comparison.pdf}
\caption{\label{fig:model1_sims} Best-fitting model predictions for a model of utterance surprisal (above) and a model of total surprisal, Eq.\ \ref{eq:total} (below).  Positive sentences are represented in purple and negative sentences in blue.  Context conditions are identified as fractions, written next to the relevant data point.  Arrows indicate data points that are not well captured by our initial model of utterance surprisal.}
\end{center} 
\end{figure}





\section{General Discussion}

What makes negation so hard? It takes longer to evaluate negative sentences than positive sentences when presented without context, but these effects are mitigated in context. We suggested a Gricean account: the processing cost of negation is related to the degree to which it violates expectations about communication in context. In our studies, by changing the proportion of people in the context who held a target item, we systematically manipulated participants' contextual expectations.  We found a parametric relationship between the strength of the context and reaction times, and this relationship was well fit by a model of the surprisal of a sentence and its referent given the context.

Previous work on sentence processing has suggested that processing negation is fundamentally difficult, perhaps due to the processing cost of negating a proposition (e.g.\ \citeNP{hclark1972}) or the cost of suppressing an affirmative representation (e.g.\ \citeNP{kaup2003}).  Our work here suggests that the difficulty of negation may not be unique to negation at all; instead, general pragmatic mechanisms could be driving this effect.  Due to the specific pragmatics of negation, negative sentences presented without context are uninformative and are thus unlikely to be produced, leading to increased surprisal and slower processing times.  In conversation, however, negative sentences are often produced when some expectation has been violated, decreasing surprisal and processing time.  

Although our specific focus was to understand the processing of negative sentences, this work has implications for sentence processing more generally.  Debates about the effects of pragmatics on linguistic processing exist in other domains (e.g.\ the processing of scalar implicatures, \citeNP{huang2009, huang2011, grodner2010}). We believe that formal models of pragmatics can provide insight into these debates and, more generally, into the role that context plays in linguistic processing. 

\section{Acknowledgments}
This material is based upon work supported by the National Science Foundation Graduate Research Fellowship. 
% Any opinion, findings, and conclusions or recommendations expressed in this material are those of the authors and do not necessarily reflect the views of the National Science Foundation.


\bibliographystyle{apacite}

\setlength{\bibleftmargin}{.125in}
\setlength{\bibindent}{-\bibleftmargin}

\bibliography{bibLibrary}

\end{document}




