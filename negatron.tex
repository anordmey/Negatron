\documentclass[man]{apa2}

\usepackage{apacite}
\usepackage{pslatex}
\usepackage{pdfsync}
\usepackage{apacite}
\usepackage{amsmath}
\usepackage{graphicx}
\usepackage{topcapt}
\usepackage{color}

%\usepackage{setspace}
%\usepackage[margin=1in]{geometry}
%
%\doublespacing

\title{A Pragmatic Account of the Processing of Negative Sentences}
\author{Ann E. Nordmeyer and Michael C. Frank}
%\affiliation{Department of Psychology, Stanford University}

%\abstract{}

%\shorttitle{Contexts of negative sentences}

\begin{document}
\maketitle

%\begin{abstract}
%Write abstract here.  
%\end{abstract}

%%%%%%%%% INTRO %%%%%%%%% 
\section{Introduction}

Language is a powerful tool that allows us to describe not only the state of the world as we see it, but also the world as it is not.  If I am a regular at a coffee shop and always order chai, but the shop has run out today, the barista might say ``We don't have any chai today'' when I enter.  Negative sentences are very informative when expectations are violated.

Although negation is critical for communicating many meanings, processing negative sentences can be slow and effortful.  In sentence verification tasks, participants who are asked to evaluate the truth of a sentence describing a picture take significantly longer to evaluate negative sentences compared to positive ones \cite{hclark1972, carpenter1975, just1971, just1976}. In these experiments, participants compare positive or negative sentences to simple pictures which either match or do not match the sentences.  Adult participants display a main effect of sentence type, taking longer to respond to negative sentences, as well as in interaction between sentence type and truth value, such that participants respond fastest to true positive sentences and \emph{slowest} to true negative sentences.  This work has been taken as evidence for a propositional view of negation, in which a negative sentence is represented as the denial of an affirmative proposition, leading to greater processing time for negative as opposed to positive sentences. 

The finding that negative sentences take longer to process than positive sentences has been replicated in other paradigms as well. Electroencephalography studies have shown that semantically unexpected/false sentences such as ``a robin is a truck'' elicit a more negative peak at the N400 compared to sentences such as ``a robin is a bird''.  However, sentences such as ``a robin is not a truck'' produce a greater N400 response than sentences such as ``a robin is not a bird'', suggesting that the negative element ``not'' is not processed immediately (\citeNP{fischler1983}, see also \citeNP{ludtke2008}).  Similar results have been found using probe-recognition tasks.  For example, the sentence ``The door was open'' primes recognition of  a matching picture (an open door) compared to a mismatching picture (a closed door) after 750 ms delay, whereas negative sentences such as ``The door was not open'' do not prime the matching picture (a closed door) until a 1000 ms delay.  (\citeNP{kaup2006}, see also \citeNP{kaup2003, hasson2006}).  Although there is some disagreement about the nature of these representations (e.g. as the denial of a proposition (see \citeNP{hclark1972}), or a mental model of the negated and actual state of affairs (see \citeNP{kaup2003}), this work collectively suggests that processing negative sentences is often difficult.

There is a critical difference, however, between evaluating a sentence in the lab and comprehending speech in the real world. According to Grice's Cooperative Principle \cite{grice1975}, speakers should produce utterances that are truthful, relevant, and informative.  Negative sentences presented without context violate this principle.  If the barista from our first example says ``we don't have chai today'' to a customer who always orders coffee, this utterance would be neither relevant nor informative.  In general, negations are produced when there is some expectation that the speaker wishes to reverse.  

Congruent with this Gricean account, a number of studies have shown that a supportive context mitigates the processing cost of negation.  Noting that denials are generally produced in response to a violation of expectations, Wason (1965) designed a study to examine whether pragmatic constraints such as context would affect how participants responded to negative sentences.  Participants viewed stimuli consisting of 8 colored dots, in which 7 dots were one color and 1 dot was a different color.  Participants were asked to describe the stimuli, and then evaluate positive and negative sentences about the stimuli.  Wason found that when participants' descriptions of the stimuli highlighted the fact that one of the dots was an exception to the rule, they were faster to evaluate negative sentences about the ``odd'' dot.  Additional work has shown that when participants read sentences that set up a supportive context, reading times for negative sentences are reduced \cite{glenberg1999}, and false positive and false negative sentences show similar N400 responses when the negative sentences are pragmatically licensed \cite{nieuwland2008}.  Children are also sensitive to context effects when processing negative sentences, responding faster to respond to negative sentences about a toy that is different from other toys in the group \cite{devilliers1975}.

Some researchers have found that some contexts are more effective than others at facilitating the processing of negative sentences.  In one such study, contexts which explicitly mentioned or strongly implied the negated characteristic were more effective at reducing reading times of negative sentences compared to contexts which did not mention the relevant characteristic \cite{ludtke2006}.  Critically, these contexts did not have a significant effect on the reading times for positive sentences.  In addition, a mouse-tracking study conducted by Dale and Duran (2011) \nocite{dale2011} found that enhanced contexts lead to significantly smoother mouse trajectories when selecting whether a sentence was true or false, but simple one-sentence contexts did not have an effect.  But although these findings are congruent with the idea that pragmatic expectations are the source of negation's processing cost, they do not directly test that hypothesis.  The goal of our current work is to make such a test.

What is the mechanism by which context influences the processing of negation?  We propose that negative sentences are more informative in contexts that set up a strong expectation that is violated. If the processing cost of negation is pragmatic, then more informative negative sentences should elicit smaller reaction times. How should we quantify informativeness in context? Recent modeling work quantifies pragmatic reasoning in simple experimental contexts \cite{frank2012,goodman2013}. The assumption underlying this work is that speakers are informative---they will produce utterances that will pick out smaller subsets of the context, leaving as little ambiguity as possible for the listener.  We use this definition of informativeness to provide a quantitative interpretation of our hypothesis.

To link informativeness---as computed in a probabilistic model---to reaction time, we assume that reaction time is proportional to \emph{surprisal}. Surprisal is an information-theoretic measure of the amount of information carried by an event (in this case, an utterance in some context) based on its probability. Surprisal has been used effectively to predict reaction times from probabilistic models \cite{levy2008}; this work provides inspiration for our current model. 

We test the hypothesis that pragmatic surprisal explains the processing cost of negative sentences. Study 1 measures this processing cost, replicating previous findings that context facilitates the processing of negation (DO I STILL WANT TO INCLUDE THIS??).  Study 2 investigates the effect of the strength of the context by parametrically varying the base rate of a negated feature. We find that adults are sensitive to very small changes in the strength of the context, and that these changes have a graded effect on the processing speed of negative sentences.  We compute the surprisal of sentences in these contexts, and find that both participants' descriptions of stimuli (Study 3) and a model of pragmatic informativeness predict the relationship between context and reaction time.  These results provide novel, quantitative support for the hypothesis that context influences negative sentence processing by modulating listeners' expectations.

\section{Study 1: Context vs. No Context}

To test whether non-linguistic contextual expectations alleviate the processing cost of negative sentences, we constructed a simple sentence verification task based on \citeA{hclark1972}.  Previous studies of the relationship between context and negation have required participants to actively engage with the context, either by describing pictures \cite{wason1965} or reading sentences \cite{glenberg1999}.  Here, participants passively viewed a visual context, providing further evidence that the effects of context on negative sentence processing are robust in a wide variety of contexts.

\subsection{Method}

\subsubsection{Participants}

We recruited 100 participants to participate in an online experiment through the Amazon's Mechanical Turk (mTurk) website.\footnote{Previous work has shown that mTurk is an effective tool for collecting RT data \cite{crump2013}.}  Participants ranged in age from 18-65; 63 were male and 37 female.  We restricted participation to individuals in the United States. We paid participants 30 cents to participate, which took approximately 5 minutes to complete.  

\subsubsection{Stimuli}

\begin{figure}[t]
\begin{center} 
\includegraphics[width=3.25in]{figures/negatron_trialfig2.pdf}
\caption{\label{fig:trial} An example trial, consisting of two separate slides (shown sequentially): a context slide and a trial slide for a true negative trial. }
\vspace{-5mm}
\end{center} 
\end{figure}

Twenty-eight trial items were created in which a character was shown holding either two of the same common, recognizable objects (e.g.\ two apples), or holding nothing.  On each trial a sentence of the form ``[NAME] [has/has no] [ITEM]'' was written.  Half of the sentences were positive and half were negative, and they were paired with pictures such that half were true and half were false.  The experiment was fully crossed, with participants receiving seven true positive, seven false positive, seven true negative and seven false negative sentences in a randomized order over the course of the study.  

Participants were randomly assigned to the ``no context'' condition or the ``context'' condition.  Participants in the no context condition saw a blank screen with a fixation cross before each trial, while participants in the context condition viewed a context slide.  The context slide showed three characters, each holding the same two identical items.  The characters all differed from the trial character and from each other in hair and shirt color.  A sentence instructed participants to ``Look at these [boys/girls]!'' (Fig.\ \ref{fig:trial}).  


\subsubsection{Procedure}
Participants were first presented with an instructions screen which described the task and informed them that they could stop at any time.  Once they accepted the task, they were given eight positive sentence practice trials with feedback about incorrect responses. 

In each trial, participants saw a context (3s) and then a picture and a sentence. They were asked to read the sentence and respond as quickly and accurately as possible with a judgment of whether it was true or false when applied to the picture.  We recorded reaction times for each trial, measured as the time from when the picture and sentence were presented to the moment when the response was made.

\subsubsection{Data Processing}
We excluded from analysis 6 participants who did not list English as their native language, 7 participants for having participated in a previous pilot study, and 4 participants for having an overall accuracy of below 80\%.  Thus, data from a total of 83 participants were analyzed.  We also excluded trials with RTs greater than 3 standard deviations from the log-transformed mean.  

\subsection{Results \& Discussion}

\begin{figure}
\begin{center} 
\includegraphics[width=3.25in]{figures/study1_linegraph.pdf}
\caption{\label{fig:e1line} Reaction times for each trial type across different conditions.  Responses to true sentences are shown on the left, and false sentences are shown on the right.  Negative sentences are shown in grey, and positive sentences in black.  Error bars show 95\% confidence intervals.}
\end{center} 
\end{figure}

Negative sentences were difficult to process when presented without context; in context, this effect disappeared (Fig.\ \ref{fig:e1line}).  This result is congruent with previous work on sentence verification, which has also found a main effect of negation \cite<e.g.>{hclark1972} and with work examining the role of context in negation \cite<e.g.>{wason1965, nieuwland2008, dale2011}.  

To examine the reliability of these findings, we fit a linear mixed-effects model to participants' reaction times.  We examined the interaction between sentence type, truth value, and context on reaction times.\footnote{All mixed-effects models were fit using the lme4 package in R version 2.15.3.  The model specification was as follows: \texttt{RT $\sim$ sentence~$\times$~truth~$\times$~context + (sentence~$\times$~truth~\textbar~subject) +  (sentence~$\times$~truth~\textbar~item)}.  Significance was calculated using the standard normal approximation to the $t$ distribution \cite{barr2013}. Data and analysis code can be found at FILL THIS IN LATER} Results of this model show a main effect of truth value, with significant faster reaction times for true sentences compared to false sentences ($\beta= -196$, $p< .001$).\footnote{Coefficient weights are interpretable in milliseconds.}  Although there was no main effect of negation across both conditions, there was an interaction between sentence type and truth value ($\beta= 260$, $p< .001$), replicating the finding that participants respond fastest to true positive sentences but slowest to true negative sentences \cite{hclark1972}.  Critically, there was a significant 3-way interaction between context condition, sentence type, and truth value ($\beta= -227$, $p< .01$), suggesting that this interaction was primarily driven by the slow RTs for true negative sentences in the no context condition.  

To understand why context had the strongest effect on true negative sentences, consider what a true negative trial looks like in the no context condition.  These are trials in which the participant has no expectation about what the character might be holding, because no context was provided to set up such an expectation.  The participant would then see a picture of an empty-handed boy with the sentence ``Bob has no apples.''  These types of trials likely cause participants to falter because there is no reason for ``apples'' to be mentioned at all.  However, when a participant first views a context such as the one in Fig.\ \ref{fig:trial}, they can form an expectation that boys typically have apples.  Now, when participants see a boy with no apples, a sentence such as ``Bob has no apples'' makes sense.

Study 1 contributes to a body of evidence suggesting that negative sentences are more felicitous when they negate an expectation, and that such expectations can be set up by an appropriate context.  In Study 2, we examine how systematically manipulating the strength of the context might produce changes in reaction times by altering the expectations created by the context.  

\section{Study 2: Varying strength of context}

Should all contexts be equally helpful in processing negation? In Study 2, we parametrically manipulated the strength of the context.  This is an extension of previous work (\cite{ludtke2006, dale2011}, which has relied on qualitative descriptions of the differences between effective and ineffective contexts for negative sentence processing.  By providing a quantitative manipulation of context, we can test explicit predictions about how the strength of the expectations set up by the context influences the processing of negative sentences.  

In Study 2, the target character (e.g. the picture that the sentence described) was embedded within the three context characters.  On each trial, participants saw four characters, and some subset of those characters were holding target items (e.g. apples).  After several seconds, a red box appeared around one of the characters, and a sentence appeared about that picture.  Participants were instructed to decide as quickly as possible whether the sentence was true or false.  We made this design change to make our stimuli more comparable to previous studies on the effect of context on sentence processing \cite{frank2012, wason1965}, and to eliminate any ``surprise'' that might arise from the appearance of the target character \footnote{Previous tests of the effects of context on negation processing, using the same design and stimuli as Study 1, found that the strongest contexts -- for example, contexts where every character in the context had had apples but the target did not -- actually led to slightly slower reaction times compared to contexts where some characters in the context had apples and some did not.  The interpretation of this finding was that seeing an unexpected picture led to a slowdown in reaction time, regardless of the sentence that accompanied it.  The results of this experiment support this interpretation.  A summary of these previous experiments can be found in \cite{nordmeyer2014}.}

Because we were interested in whether the probability of a certain description would influence participants' reaction times, we made every character identical except for the presence or absence of target items.  This allowed us to focus specifically on the probability or a positive or negative sentence, constraining participants' hypotheses about the sentence that would appear.  If the context gives participants a glimpse into the ``world'' that each trial exists in, this represents a small sample of the base rate of what the characters in this world look like.  By manipulating this base rate, we can change peoples' expectations about the trial character.  If the differences in reaction times between the no context and the context condition in Study 1 are due to the relative informativeness of the negative utterance based on the context, we should expect to see a relationship between the strength of the context and reaction time. 

\subsection{Method}

\subsubsection{Participants} 

We again recruited ??? participants from mTurk (??? male and ??? female, ages ?? -- ??). We restricted participation to individuals in the US and paid 50 cents for this 10 minute study.  

\subsubsection{Stimuli}

FIXME: ADD STIMULI FIGURE

Thirty-two trial items were created in which a character was shown holding either two of the same common, recognizable objects (e.g.\ two apples), or holding nothing.  Within each trial, the character holding items was identical to the character with no items in all other respects.  A within-subjects factor determined what type of context participants saw on each trial.  The context condition determined the proportion of characters other than the target character who were holding target items.  Context conditions showed $\frac{0}{3}$, $\frac{1}{3}$, $\frac{2}{3}$, or $\frac{3}{3}$ of the context characters holding objects. The order of characters was shuffled on each trial, so participants did not know which character was the target picture until a read box appeared around it.  

On each trial a sentence of the form ``[NAME] [has/has no] [ITEM]'' appeared.  Half of the sentences were positive and half were negative, and they were paired with pictures such that half were true and half were false.  The experiment was fully crossed, with participants receiving eight true positive, eight false positive, eight true negative and eight false negative sentences distributed equally across context types in a randomized order over the course of the study.  

\subsubsection{Procedure}
Participants were first presented with an instructions screen which described the task and informed them that they could stop at any time.  Once they accepted the task, they were given eight positive sentence practice trials with feedback about incorrect responses. 

In each trial, participants saw an array of four pictures: The target pictures and three context pictures presented in a random order.  Participants were told to look at these pictures for four seconds, at which point a red box appeared around one of the pictures.  One second later, a sentence about that picture appeared.  Participants were told to read the sentence and respond as quickly and accurately as possible with a judgment of whether it was true or false when applied to the highlighted picture.  We recorded reaction times for each trial, measured as the time from when the sentence was presented to the moment when the response was made.

 
 \subsubsection{Data Processing}
We excluded ?? participants who did not list English as their native language, ?? participants for participating in a previous iteration of the experiment, and ?? participants for having an overall accuracy below 80\%.  Thus, we analyzed data from a total of ?? participants. As in Study 1, we only analyzed correct trials and excluded trials with RTs greater than 3 SDs from the log-transformed mean. 

Because we were interested in the effect of context, results from these two studies were combined and analyzed together with context condition re-coded as a continuous variable by calculating the proportion of people in each context condition who had a target item (e.g.\ the $\frac{1}{3}$ condition in Study 2a was recoded as .33). ((FIXME: DO I WANT TO DO THIS DIFFERENTLY?)

\subsection{Results and Discussion}


FIXME: NEW DATA
\begin{figure}[t]
\begin{center} 
\includegraphics[width=3.25in]{figures/study2a_linegraph.pdf}
\includegraphics[width=3.25in]{figures/study2b_linegraph.pdf}
\caption{\label{fig:e2line} Reaction times for each trial type across different conditions. Responses to true sentences are shown on the left, and false sentences are shown on the right.  Negative sentences are shown in grey, and positive sentences in black.  Data for Study 2a (3-person contexts) are shown above, and data for Study 2b (4-person contexts) are shown below.  The context condition is notated by a fraction representing the number of characters in the context who held target items. Error bars show 95\% confidence intervals.  }
\end{center} 
\end{figure}

As the proportion of target items in the context increased, reaction times tended to decrease, particularly for negative and false sentences, supporting our hypothesis (Fig.\ \ref{fig:e2line}).  

We fit a linear mixed-effects model to reaction times in response to sentences.  We examined the interaction between sentence type, truth value, and context on reaction times.\footnote{FIXME: CHANGE THIS.  The model specification was as follows: \texttt{RT $\sim$ sentence~$\times$~truth~$\times$~context + (sentence~$\times$~truth~\textbar~subject) +  (sentence~$\times$~truth~\textbar~item)}.}  

Quantitative manipulation of the strength of the context resulted in systematic changes in the processing cost of negation, particularly for true negative sentences.  This finding is consistent with our initial hypothesis: As the proportion of people in the context with the target item increases, describing the trial picture as \emph{not} having that target item becomes more informative.  That is, the more people in the context who have apples, the more we expect a person with nothing to be described as ``a boy with no apples.'' 

FIXME: NEED MORE HERE.  

\section{Study 3}

In Studies 1 and 2, we demonstrated that a simple visual context can facilitate the processing of negative sentences, and that there is a linear relationship between the strength of the context and the effect on reaction time to evaluate negative sentences.  Contexts that set up a strong expectation lead to faster reaction times to process a negative sentence when that expectation is violated.  Our prediction was based on previous work demonstrating that participants expect speakers to be informative when speaking \cite{frank2012}.  If everyone in the context has a specific feature, and the trial character is lacking that feature, it is highly informative to describe the trial character in terms of the negation of the expected feature.  However, although the results of Studies 1 and 2 support this interpretation, we do not know for sure whether participants expect speakers to use negation in these contexts.  Study 3 attempts to directly measure participants' expectations of how a speaker would describe the pictures seen in Studies 1 and 2, depending on the context.

\subsection{Method}

\subsubsection{Participants} 

We again recruited ??? participants from mTurk (??? male and ??? female, ages ?? -- ??). We restricted participation to individuals in the US and paid 50 cents for this 10 minute study.  

\subsubsection{Stimuli}

FIXME: ADD STIMULI FIGURE

Stimuli were identical to those used in Study 2, with the same within-subjects context condition.  The only difference was that after the target item was highlighted in red, an incomplete sentence appeared (e.g. ``[NAME] has $\rule{3cm}{0.15mm}$.'').  In half of the trials, the highlighted picture was holding target items, and in half of the trials, the highlighted picture was holding nothing.  The experiment was fully crossed such that target characters appeared with or without target items an equal number of times in each context type.  

\subsubsection{Procedure}

Participants were first presented with an instructions screen which described the task and informed them that they could stop at any time.  

In each trial, participants saw an array of four pictures: The target pictures and three context pictures presented in a random order.  Participants were told to look at these pictures for four seconds, at which point a red box appeared around one of the pictures.  One second later, an incomplete sentence appeared.  Participants were told to finish the sentence (by typing into a small text box) using only a few words, in a way that would help someone else identify the character in the red box if they saw the pictures in a different order.  
 
 \subsubsection{Data Processing}
We excluded ?? participants who did not list English as their native language and ?? participants for participating in a previous iteration of the experiment.  Thus, we analyzed data from a total of ?? participants.  

\subsection{Results and Discussion}


FIXME: NEW GRAPHS
\begin{figure}[t]
\begin{center} 
\includegraphics[width=3.25in]{figures/study2a_linegraph.pdf}
\includegraphics[width=3.25in]{figures/study2b_linegraph.pdf}
\caption{\label{fig:e2line} Reaction times for each trial type across different conditions. Responses to true sentences are shown on the left, and false sentences are shown on the right.  Negative sentences are shown in grey, and positive sentences in black.  Data for Study 2a (3-person contexts) are shown above, and data for Study 2b (4-person contexts) are shown below.  The context condition is notated by a fraction representing the number of characters in the context who held target items. Error bars show 95\% confidence intervals.  }
\end{center} 
\end{figure}

FIXME: WHAT RESULTS HERE?


In Studies 1 and 2, we found that including a visual context facilitates the processing of negative sentences, and that this effect is modulated by the strength of the expectation set up by the context.  As the proportion of people with a target item in the context increases, the reaction time to process a negative sentence that negates that target property decreases.  We hypothesized that this effect is due to the ways that context changes expectations about what a speaker might say to describe a picture.  If you see a context in which the every character except for one is holding apples, it makes sense to predict that the boy holding nothing might be described as a ``boy with no apples''.  However, if you see a context in which nobody is holding anything, it would be odd to describe one of them as ``a boy with no apples'', because you have no reason to expect that he \emph{would} have apples.  Our prediction was that these expectations, calculated as the surprisal of seeing a picture described using a certain sentence, would be positively correlated with reaction times from previous studies.  Study 3 confirmed this hypothesis.  

\section{Model}

Studies 1 and 2 show that a simple visual context can facilitate the processing of negation, with contexts that set up a strong expectation leading to faster RTs for negative sentences.  In Study 3, we used participants' spontaneous descriptions of our stimuli to calculate the surprisal of different sentence types, and demonstrated that participants' reaction times are highly correlated with the surprisal of an utterance. 

We hypothesized that this effect was driven by the expectation that speakers are informative \cite{grice1975,frank2012}: If everyone in a context has a specific feature, and the target character is lacking that feature, it is highly informative to describe the target character in terms of the negation of the expected feature. In contrast, if no one has a feature, it's pragmatically odd to negate it. In this section, we formalize these intuitions. Due to the Gricean nature of the intuition---which lead us to consider a truthful speaker as well---we focus here on predicting the processing of true sentences.   

We modeled the behavior of participants in our experiments by assuming that reaction time is proportional to the surprisal of the utterance $w$, given the context $C$ and the speaker's intended referent $r_S$ (following \citeNP{levy2008}):

\begin{equation}\label{eq:surprise}
RT \sim -\log(P(w| r_s, C)).
\end{equation}

\noindent We then define the probability of the utterance as proportional to its utility (following \citeNP{frank2012}):

\begin{equation}\label{eq:pw1}
P(w | r_s, C) \propto  e^{U(w;r_s,C)},
\end{equation} 

\noindent This utility is defined as the informativeness of $w$ minus its cost $D(w)$:

\begin{equation}\label{eq:utility}
U(w;r_s,C) = I(w;r_s, C) - D(w).
\end{equation}

\noindent Informativeness in context is calculated as the number of bits of information conveyed by the word. We assume that $w$ has a uniform probability distribution over its extension in context (e.g.\ ``boy with apples'' applies to any boy who has apples, leading to a probability of $1/|w|$ of picking out each individual boy with apples) :

\begin{equation}\label{eq:info}
I(w;r_s, C) = -(-\log(|w|^{-1})).
\end{equation}

\noindent The cost term $D(w)$ can then be defined in any number of ways; in this model we define it as the number of words in the utterance multiplied by a cost-per-word parameter.  Note that in our experiment, the negative sentences always have exactly one word more than the positive sentences. 

We created a sparse vocabulary which represented possible words to describe the characters.  This included the target utterance (e.g.\ ``apples'' and ``no apples''), as well as words that were uniformly true or false of all characters. Combining Equations \ref{eq:pw1}--\ref{eq:info}, and normalizing Eq.\ \ref{eq:pw1} over all possible words in the vocabulary $V$, we have:

\begin{equation}\label{eq:pw2}
P(w | r_s, C) = \frac{ e^{\log(|w|^{-1}) - D(w)}} {\sum_{w' \in V}{e^{\log(|w'|^{-1}) - D(w')}}}.
\end{equation}

\noindent Combining Eq. \ref{eq:surprise} with Eq. \ref{eq:pw2}, this model predicts that as the number of e.g.\ boys with apples in the context increases, the informativeness of the negative sentence ``Bob has no apples'' increases, because it selects an increasingly smaller subset of the context. Highly informative sentences will have high probability, hence lower surprisal and faster RTs. 


FIXME: UPDATE WITH NEW DATA.
We fit this model to data from Study 2a, with cost = 3 (Table 1).  When the model was fit to the combined data from Studies 2a and 2b, the cost-per-word parameter remained the same (Fig.\ \ref{fig:model1_sims}).  This model accounted for a substantial amount of variance in participant reaction times from Study 2 ($r=.76$, $p<.001$).  Nevertheless, the model fails to capture the U-shaped relationship seen in Study 2; specifically, it underestimates the surprisal of $\frac{0}{3}$ and $\frac{0}{4}$ contexts for positive sentences, and $\frac{3}{3}$ and $\frac{4}{4}$ contexts for negative sentences.In these trials, participants may have found the target picture surprising regardless of the sentence that they read. For example, in $\frac{0}{3}$ and $\frac{0}{4}$ contexts followed by a true positive trial, participants saw several boys with nothing, and then saw a boy holding something.  


\section{General Discussion}

What makes negation so hard? It takes longer to evaluate negative sentences than positive sentences when presented without context, but these effects are mitigated in context. We suggested a Gricean account: the processing cost of negation is related to the degree to which it violates expectations about communication in context. In our studies, by changing the proportion of people in the context who held a target item, we systematically manipulated participants' contextual expectations.  We found a parametric relationship between the strength of the context and reaction times, and this relationship was well fit by a model of the surprisal of a sentence and its referent given the context.

Previous work on sentence processing has suggested that processing negation is fundamentally difficult, perhaps due to the processing cost of negating a proposition (e.g.\ \citeNP{hclark1972}) or the cost of suppressing an affirmative representation (e.g.\ \citeNP{kaup2003}).  Our work here suggests that the difficulty of negation may not be unique to negation at all; instead, general pragmatic mechanisms could be driving this effect.  Due to the specific pragmatics of negation, negative sentences presented without context are uninformative and are thus unlikely to be produced, leading to increased surprisal and slower processing times.  In conversation, however, negative sentences are often produced when some expectation has been violated, decreasing surprisal and processing time.  

Although our specific focus was to understand the processing of negative sentences, this work has implications for sentence processing more generally.  Debates about the effects of pragmatics on linguistic processing exist in other domains (e.g.\ the processing of scalar implicatures, \citeNP{huang2009, huang2011, grodner2010}). We believe that formal models of pragmatics can provide insight into these debates and, more generally, into the role that context plays in linguistic processing. 

FIXME: I probably need more here.  

\section{Acknowledgments}
This material is based upon work supported by the National Science Foundation Graduate Research Fellowship. 
% Any opinion, findings, and conclusions or recommendations expressed in this material are those of the authors and do not necessarily reflect the views of the National Science Foundation.


\bibliographystyle{apacite}

\setlength{\bibleftmargin}{.125in}
\setlength{\bibindent}{-\bibleftmargin}

\bibliography{bibLibrary}

\end{document}




